% STEP 1: Choose oneside or twoside. Use the 'draft' option a lot when writing.
\documentclass[english, oneside]{HYgradu}

\usepackage[utf8]{inputenc} % For UTF8 support. Use UTF8 when saving your file.
\usepackage{lmodern} % Font package
\usepackage{textcomp}
\usepackage[pdftex]{color, graphicx} % For pdf output and jpg/png graphics
\usepackage[pdftex, plainpages=false]{hyperref} % For hyperlinks and pdf metadata
\usepackage{fancyhdr} % For nicer page headers
%\usepackage{tikz} % For making vector graphics (hard to learn but powerful)
%\usepackage{wrapfig} % For nice text-wrapping figures (use at own discretion)
\usepackage{amsmath, amssymb} % For better math
%\usepackage[round]{natbib} % For bibliography
\usepackage[footnotesize,bf]{caption} % For more control over figure captions

\fussy % Probably not needed but you never know...

% OPTIONAL STEP: Set up properties and metadata for the pdf file that pdfLaTeX makes.
% But you don't really need to do this unless you want to.
\hypersetup{
    bookmarks=true,         % show bookmarks bar first?
    unicode=true,           % to show non-Latin characters in Acrobat’s bookmarks
    pdftoolbar=true,        % show Acrobat’s toolbar?
    pdfmenubar=true,        % show Acrobat’s menu?
    pdffitwindow=false,     % window fit to page when opened
    pdfstartview={FitH},    % fits the width of the page to the window
    pdftitle={},            % title
    pdfauthor={},           % author
    pdfsubject={},          % subject of the document
    pdfcreator={},          % creator of the document
    pdfproducer={pdfLaTeX}, % producer of the document
    pdfkeywords={something} {something else}, % list of keywords for
    pdfnewwindow=true,      % links in new window
    colorlinks=true,        % false: boxed links; true: colored links
    linkcolor=black,        % color of internal links
    citecolor=black,        % color of links to bibliography
    filecolor=magenta,      % color of file links
    urlcolor=cyan           % color of external links
}

% STEP 2:
% Set up all the information for the title page and the abstract form.
% Replace parameters with your information.
\title{Your Title Here}
\author{Your Name}
\date{\today}
\level{Master's thesis}
\faculty{Faculty of Whatever}
\department{Department of Something}
\address{PL 42 (Kuvitteellinen katu 1)\\00014 Helsingin yliopisto}
\subject{Your Field}
\prof{prof. Smith}
\censors{prof. Smith}{doc. Smythe}{}
\depositeplace{}
\additionalinformation{}
\numberofpagesinformation{\numberofpages\ pages}
\classification{}
\keywords{Your keywords here}
\quoting{``Bachelor's degrees make pretty good placemats if you get them laminated.'' \\---Jeph Jacques}

\begin{document}

% Generate title page.
\maketitle

% STEP 3:
% Write your abstract (of course you really do this last).
% You can make several abstract pages (if you want it in different languages),
% but you should also then redefine some of the above parameters in the proper
% language as well, in between the abstract definitions.
\begin{abstract}
Abstract goes here.
\end{abstract}

% Place ToC
\mytableofcontents



% -----------------------------------------------------------------------------------
% STEP 4: Write the thesis.
% Your actual text starts here. You shouldn't mess with the code above the line except
% to change the parameters. Removing the abstract and ToC commands will mess up stuff.
\chapter{Introduction}

Lorem ipsum dolor sit amet, consectetur adipiscing elit. Fusce vel sem et nisl consectetur
porta vitae eget diam. Donec consequat magna vitae nisi tristique dictum id non ipsum.
Mauris non mi porta dolor ullamcorper tincidunt. Ut volutpat est nec mi aliquet auctor
eleifend nisl ornare. Donec dapibus nunc at purus fermentum pretium laoreet justo
fringilla. Nulla pellentesque vulputate nisl eu bibendum. Lorem ipsum dolor sit amet,
consectetur adipiscing elit. Integer mattis libero non erat rhoncus tincidunt. Etiam quis
ante ut magna feugiat venenatis. Mauris cursus massa non augue aliquam mollis. Fusce
dictum, ante quis dapibus scelerisque, velit leo faucibus est, laoreet convallis eros
ligula eget orci. Phasellus lobortis tortor id sapien commodo elementum. Sed sed egestas
dolor. Nulla massa ligula, dignissim sed auctor a, ullamcorper ac velit. Vivamus porttitor
tempus bibendum. Lorem ipsum dolor sit amet, consectetur adipiscing elit.


% STEP 5:
% Uncomment the following lines and set your .bib file and desired bibliography style
% to make a bibliography with BibTeX.
% Alternatively you can use the thebibliography environment if you want to add all
% references by hand.

%\clearpage
%\addcontentsline{toc}{chapter}{Bibliography} % This lines adds the bibliography to the ToC
%\bibliographystyle{plain}
%\bibliography{foo.bib}


\end{document}

